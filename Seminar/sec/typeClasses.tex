\begin{frame}{Introduction: Type classes}
	When we want to extend an existing class with a new interface 
	(and so, possibly, with new functionalities) we have essentially four choices:	
	\begin{enumerate}[<+->]
		\item Extending the class itself (not always possible, e.g. API classes)
		\item Creating a new class that extends the original class and the new interface (not always possible, e.g. final classes)
		\item Creating a wrapping class that contains the original class and extends the new interface
		\item Using type classes
	\end{enumerate}
	\pause[4]
	The last choice is a classical pattern of FP.			
\end{frame}

\begin{frame}[fragile]{Introduction: Type classes - Example (1)}
	Let's introduce type classes with an example.
	We have this scala interface
\begin{lstlisting}[language=scala]
trait Summable[T] {
	def sumElements(list: List[T]): T
}
\end{lstlisting}
	and we want Int and String classes to implement it.

	\pause

	Of course we cannot change Int or String classes directly (they are in scala's API), so the first option is discarded.
	
	\pause
	
	We could create new classes MyInt and MyStr that extend respectively Int and String but those classes are final, so we cannot work in 
	this way.
	
	\pause
	
	Surely we can create wrapping classes MyInt and MyStr, but in this way we have now two definitions of Int and of String and 
	it could get messy.
\end{frame}

\begin{frame}[fragile]{Introduction: Type classes - Example (2)}	
	Let's try type classes!
	We start creating two implicit objects that implement the interface as follows:
\begin{lstlisting}[language=scala]
implicit object IntSummable extends Summable[Int] {
	def sumElements(list: List[Int]): Int = list.sum
}
implicit object StringSummable extends Summable[String] {
	def sumElements(list: List[String]): String = {
		list.mkString("")
	}
}
\end{lstlisting}

	\pause

	In this way we can now define a generic method that uses sumElements
\begin{lstlisting}[language=scala]
def sumAll[T](l: List[T])(implicit summable: Summable[T]): T = summable.sumElements(l)		
\end{lstlisting}	
\end{frame}

\begin{frame}[fragile]{Introduction: Type classes - Example (3)}
	The advantage of this approach is that we can use sumAll method only with Int and String (all the other types yield
	a compile error) and, because of implicit, we don't have to pass summable object to sumAll.
	
\begin{lstlisting}[language=scala]
sumAll(List(1,2,3)) // 6
sumAll(List("Scala's ", "awesome")) // "Scala's awesome"
sumAll(List(true, true, false)) // COMPILE ERROR				
\end{lstlisting}

\end{frame}
	
\begin{frame}[fragile]{Introduction: Type classes - Example (4)}
	
	Through extension method, we could also use sumAll as method in List of Int and List of String:
\begin{lstlisting}[language=scala]
implicit class ListSummable[A](l: List[A])(implicit summable: Summable[A]) {
	def sumAll: A = summable.sumElements(l)
}
\end{lstlisting}	
	
	\pause
	
\begin{lstlisting}[language=scala]
List(1,2,3).sumAll //6
List("Scala ", "is ", "awesome").sumAll //"Scala is awesome"
List(true, true, false).sumAll // COMPILE ERROR
\end{lstlisting}
\end{frame}