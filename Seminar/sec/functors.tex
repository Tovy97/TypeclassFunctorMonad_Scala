\begin{frame}[fragile]{Functors}
	Let's start with the map function:
\begin{lstlisting}[language=scala]
def map[A, B](fa: F[A])(f: A => B): F[B]
\end{lstlisting}
	\pause
	\begin{block}{Definition}
		A functor is a class that has a map method that respects the following laws:
		\begin{enumerate}[<+->]
			\item \textbf{Identity:} given the identity function $id$ and a functor $obj$ we have that $map(obj)(id) = obj$.
			\item \textbf{Composition:} given two functions $f$ and $g$ and a functor $obj$ we have that $map(map(obj)(f))(g) = map(obj)(g \circ f)$
		\end{enumerate}
	\end{block}	
\end{frame}

\begin{frame}[fragile]{Functors - Implementation (1)}	
	We'll use type classes to implement functors.
	
	Functors can be represented with the following interface
\begin{lstlisting}[language=scala]
trait Functor[F[_]] {
	def map[A, B](fa: F[A])(f: A => B): F[B]
}
\end{lstlisting}
\end{frame}

\begin{frame}[fragile]{Functors - Implementation (2)}			
	Let's create List and Option functors.
\begin{lstlisting}[language=scala]
implicit object listFunctor: Functor[List] {
	def map[A, B](fa: List[A])(f: A => B): List[B] = fa.map(f)
}
implicit object optionFunctor: Functor[Option] {
	def map[A, B](fa: Option[A])(f: A => B): Option[B] = 
		fa match {
			case None => None
			case Some(x) => Some(f(x))	
		}
}			
\end{lstlisting}		
	The first case is easy because List class has already a map function
	(honestly, Option has a map method too).
\end{frame}
	
\begin{frame}[fragile]{Functors - Implementation (3)}			
	As before, we can create a generic method that applies map:
\begin{lstlisting}[language=scala]
def applyMap[F[_], A, B](a: F[A])(f: A => B)(implicit functor: Functor[F]): F[B] = functor.map(a)(f)
applyMap(List(1, 2, 3))(_ + 1) //List(2, 3, 4)
applyMap(Option(1))(_ + 1) // Some(2)
applyMap(Set(true, true, false))(!_) // COMPILE ERROR
\end{lstlisting}	
	Through extension method, we could also use applyMap as method of functors:
\begin{lstlisting}[language=scala]
implicit class ToFunctor[F[_], A, B](a: F[A])(implicit functor: Functor[F]) {
	def applyMap(f: A => B): F[B] = functor.map(a)(f)
}
List(1, 2, 3).applyMap(_ + 1) //List(2, 3, 4)
Option(1).applyMap(_ + 1) // Some(2)
Set(true, true, false).applyMap(!_) // COMPILE ERROR
\end{lstlisting}	
\end{frame}

\begin{frame}[fragile]{Functors Laws}
	Functors laws are expressed in scala in this way:
\begin{lstlisting}[language=scala]
def identity(fa: F[A]): Boolean = {
	def id(x: A): A = x	
	fa.applyMap(id) == fa
}
def composition(fa: F[A])(f: A => A, g: A => A): Boolean = {
	fa.applyMap(f).applyMap(g) == fa.applyMap(g compose f)
}
\end{lstlisting}	
	We could test these laws through PBT ScalaCheck.
\end{frame}

\begin{frame}[fragile]{Why are functors useful?}
	Functors allow to avoid boilerplate code.
	Let's take the Option type. 
	Normally before applying a function on Option we should check if that value is either Some or None through pattern matching:
\begin{lstlisting}[language=scala]
val a : Option[Int] = Some(20)
def f(a : Int) : Int = if (a % 2 == 0) a / 2 else a - 1
val ris = (a match {
	case None => None
	case Some(b) => Some(f(b))
}) //Some(10)
\end{lstlisting}	
	
	Functors reduce code:
\begin{lstlisting}[language=scala]
val a : Option[Int] = Some(10)
def f(a : Int) : Int = if (a % 2 == 0) a / 2 else a - 1
val ris = a.applyMap(f) //Some(10)
\end{lstlisting}
\end{frame}