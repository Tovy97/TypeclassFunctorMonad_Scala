\begin{frame}[fragile]{Monads}
	Again, let's start from the bind (also called flatMap) and unit methods:
\begin{lstlisting}[language=scala]
def bind[A, B](fa: M[A])(f: A => M[B]): M[B]
def unit[A](a: A): M[A]
\end{lstlisting}
	\pause
	\begin{block}{Definition}
		A monad is a functor that has bind and unit methods that respect the following laws:
		\begin{enumerate}[<+->]
			\item \textbf{Left identity:}  given a function $f$ and an object $obj$ we have that $bind(unit(obj))(f) = f(obj)$.
			\item \textbf{Right identity:} given a monad $obj$ we have that $bind(obj)(unit) = obj$
			\item \textbf{Associativity:} given two functions $f$ and $g$ and a monad $obj$ we have that 
			$bind(bind(obj)(f))(g) = bind(obj)(x \Rightarrow flatMap(f(x))(g))$
		\end{enumerate}
	\end{block}
\end{frame}

\begin{frame}[fragile]{Monads - Implementation (1a)}	
	We'll use type classes to implement monads.
	
	Monads can be represented with the following interface
\begin{lstlisting}[language=scala]
trait Monad[M[_]] {
	def bind[A, B](fa: M[A])(f: A => M[B]): M[B]
	def unit[A](a: A): M[A]
}			
\end{lstlisting}
\end{frame}

\begin{frame}[fragile]{Monads - Implementation (1b)}	
	Monads are functors, so they have $map$ function.
	Often monads are presented via another method: $join$.
	There are standard implementations of $map$ and $join$ through $bind$ and $unit$:
\begin{lstlisting}[language=scala]
trait Monad[M[_]] extends Functor[M] {
	def bind[A, B](fa: M[A])(f: A => M[B]): M[B]
	def unit[A](a: A): M[A]
	final def join[A](ma: M[M[A]]): M[A] = 
		bind(ma)(m => m)
	override final def map[A, B](fa: M[A])(f: A => B): M[B] = 
		bind(fa)(a => unit(f(a)))
}			
\end{lstlisting}
	It's possible to demonstrate that $bind(obj)(f)$ corresponds to $join(map(obj)(f))$.
\end{frame}

\begin{frame}[fragile]{Monads - Implementation (2)}	
	Let's create the List and Option monads.
\begin{lstlisting}[language=scala]
implicit object ListMonad extends Monad[List] {
	override def bind[A, B](fa: List[A])(f: A => List[B]): List[B] = fa.flatMap(f)
	override def unit[A](a: A): List[A] = List(a)
}
implicit object OptionMonad extends Monad[Option] {
	override def bind[A, B](fa: Option[A])(f: A => Option[B]): Option[B] = fa match {
			case None => None
			case Some(a) => f(a)
		}
	override def unit[A](a: A): Option[A] = Some(a)
}		
\end{lstlisting}		
	The first case is easy because List class has already a flatMap function
	(honestly, Option has a flatMap method too).
\end{frame}

\begin{frame}[fragile]{Monads - Implementation (3)}		
	As before, we can create a generic method that applies bind:
\begin{lstlisting}[language=scala]
def bind[M[_], A, B](a: M[A])(f: A => M[B])(implicit monad: Monad[M]): M[B] = monad.bind(a)(f)
bind(List(1,2,3))((x : Int) => List(x + 1)) //List(2, 3, 4)
bind(Option(1))((x : Int) => Some(x + 1)) // Some(2)
bind(Set(true, true, false))((x : Boolean) => Set(!x)) // COMPILE ERROR
\end{lstlisting}	
\end{frame}

\begin{frame}[fragile]{Monads - Implementation (4)}	
	Through extension method, we could also use monads function as method in monads:
\begin{lstlisting}[language=scala]
implicit class ToMonad1[M[_], A, B](a: M[A])(implicit monad: Monad[M]) {
	lazy val applyUnit: A => M[A] = monad.unit	
	def applyBind(f: A => M[B]): M[B] = monad.bind(a)(f)	
	def applyMap(f: A => B): M[B] = monad.map(a)(f)
}
implicit class ToMonad2[M[_], A](a: M[M[A]])(implicit monad: Monad[M]) {
	lazy val applyJoin: M[A] = monad.join(a)
}
List(1,2,3).bind((x : Int) => List(x + 1)) //List(2, 3, 4)
Option(1).bind((x : Int) => Some(x + 1)) // Some(2)
Set(true, true, false).applyBind((x : Boolean) => Set(!x)) // COMPILE ERROR	
\end{lstlisting}	
\end{frame}

\begin{frame}[fragile]{Monads Laws}
	Monads laws are expressed in scala in this way:
\begin{lstlisting}[language=scala]
def leftIdentity(a: A)(f: A => M[A]): Boolean = 
	monad.unit(a).applyBind(f) == f(a)
def rightIdentity(t: M[A]): Boolean = 
	t.applyBind(t.applyUnit) == t
def associativity(t: M[A])(f: A => M[A], g: A => M[A]): Boolean = {
	t.applyBind(f).applyBind(g) == t.applyBind((x: A) => f(x).applyBind(g))
}
\end{lstlisting}	
	We could test these laws through PBT ScalaCheck.
\end{frame}

\begin{frame}[fragile]{Differences between Functors and Monads (1)}
	Why should we use monads instead of functors?
	
	Let's start from a simple example: concatenation of map function.
\begin{lstlisting}[language=scala]
val a : Option[Int] = Some(20)
val temp = a.applyMap(_ + 1) //Some(22)
temp.applyMap(_ + 1) // Some(23)
\end{lstlisting}				
	This example works nicely. We start from an Option of Int, after the first step we have an Option of Int on which we apply again 
	the same map function and we finally get an Option of Int.
\end{frame}

\begin{frame}[fragile]{Differences between Functors and Monads (2)}
	But, let's try with a new function:
\begin{lstlisting}[language=scala]
val a : Option[Int] = Some(20)		
def f(x:Int) : Option[Int] = if (x == 0) None else Some(2/x)
val temp1 = a.applyMap(f) // return Some(Some(0))
temp1.applyMap(f)// COMPILE ERROR
\end{lstlisting}	
	Here we are stuck because, after the first step, we got an Option of Option of Int. 
	We could manage this problem and adjust the second function	to work with Option of Option of Int, 
	but in this way we would get an Option of Option of Option of Int as result type and we could continue getting more and more layers of Option.
	
	\pause
	
	Here monads come to help!
\begin{lstlisting}[language=scala]
val temp = a.bind(f) // return Some(0)
temp.bind(f) //None
\end{lstlisting}	
	We could continue to apply bind endlessly, without change to the return type.
\end{frame}

\begin{frame}[fragile]{Functors and Monads in Scala (1)}
	Scala official API doesn't contain a Functor and Monad trait, but cites them along the documentation.
	For example, in the Option class documentation we could find:
	\begin{quotation}
		The most idiomatic way to use an $Option$ instance is to treat it as a collection or \textbf{monad} and use $map$, $flatMap$\footnote{in scala $bind$ is called $flatMap$} [\dots]
	\end{quotation}	
\end{frame}

\begin{frame}[fragile]{Functors and Monads in Scala (2)}
  	Scala offers syntactic sugar to simplify call to $map$ and $flatMap$
  	methods through $for-expression$.
\begin{lstlisting}[language=scala]
val temp = for {
	i <- Option(1)
	j <- Option(2)
	k <- Option(3)
} yield i + j + k //Some(6)	
\end{lstlisting}	
	is equivalent to
\begin{lstlisting}[language=scala]
 val temp: Option[Int] = Some(1).flatMap { 
	(i: Int) => Some(2).flatMap { 
		(j: Int) => Some(3).map {
			(k: Int) => i + j + k
		}
	}
} //Some(6)	
\end{lstlisting}		
\end{frame}